%%%%%%%%%%%%%%%%%%%%%%%%%%%%%%%%%%%%%%%%%
% Freeman Curriculum Vitae
% XeLaTeX Template
% Version 2.0 (19/3/2018)
%
% This template originates from:
% http://www.LaTeXTemplates.com
%
% Authors:
% Vel (vel@LaTeXTemplates.com)
% Alessandro Plasmati
%
% License:
% CC BY-NC-SA 3.0 (http://creativecommons.org/licenses/by-nc-sa/3.0/)
%
%!TEX program = xelatex
% NOTICE: This template must be compiled with XeLaTeX, the line above should
% ensure this happens automatically but if it doesn't you will need to specify 
% XeLaTeX as the engine in your editor or script
% 
%%%%%%%%%%%%%%%%%%%%%%%%%%%%%%%%%%%%%%%%%

%----------------------------------------------------------------------------------------
%	PACKAGES AND OTHER DOCUMENT CONFIGURATIONS
%----------------------------------------------------------------------------------------

\documentclass[11pt]{article} % Font size, can be: 10pt, 11pt or 12pt

\input{structure.tex} % Include the file that specifies the document structure

% Headers and footers can be added with the \lhead{} \rhead{} \lfoot{} \rfoot{} commands
% Example right footer:
%\rfoot{\color{headings}{\sffamily Last update: \today. Typeset with Xe\LaTeX}}

%----------------------------------------------------------------------------------------

\begin{document}

\begin{paracol}{2} % Begin the multi-column environment

%----------------------------------------------------------------------------------------
%	NAME AND CURRICULUM VITAE TEXT
%----------------------------------------------------------------------------------------

\parbox[top][0.12\textheight][c]{\linewidth}{ % Parbox to hold the author name and CV text; fixed height to match the coloured box to the right, centred vertically and full line width
	\vspace{-0.03\textheight} % Reduce whitespace above the parbox to separate it from the main content
	\centering % Centre text
	{\sffamily\HUGE Jake Foster}\\\medskip % Your name
	{\color{headings}\scshape\Large\raggedright Final year student - University of Birmingham}
}

%----------------------------------------------------------------------------------------
%	MAJOR RESEARCH PROJECT
%----------------------------------------------------------------------------------------

\section{About Me}

I am a University of Birmingham student studying in my 4th year.
I have a strong passion for computer science and love to keep up to date with the latest developments in the industry.
I pride myself in being an active member of the computer science community, continuously trying to engage with a wider audience to create an environment people want to be a part of.
Problem-solving has always been a passion of mine and my determination to not let a problem go unsolved drives me to
complete all my programs to a high standard.

\medskip % Extra whitespace before the next section

%----------------------------------------------------------------------------------------
%	WORK EXPERIENCE
%----------------------------------------------------------------------------------------

\section{Work Experience}

% Blank \workposition command to add another job:

%\workposition{} % Duration
%{} % FT/PT (full time or part time)
%{} % Employer
%{} % Job title
%{} % Description

% All 5 parameters must be supplied but any can be empty if you don't need them

%------------------------------------------------
\workposition{Advanced Research Computing Assistant} % Job title
{University of Birmingham} % Employer
{07/2019 -- 08/2020} % Duration
{Birmingham, UK} % Location
{Full Time} % FT/PT (full time or part time)
{Over the first half of my placement, I created \href{https://bear-apps.bham.ac.uk}{bear-apps.bham.ac.uk}.
This was a Django application that would automatically update the list of installed applications
when new applications were added to the High Performance Computing cluster.
Most of the text was scrapped from a different set of pages on the website using BeautifulSoup, a Python library designed
to obtain and parse HTML dumps.


For the second half of my placement, I added Centos 8 support to the HPC cluster.
All the configs were written in SaltStack. 
I had to code nftables configs to replace the soon-to-be-deprecated iptables, sssd to replace nslcd
and fix general package config problems caused by the OS upgrade (For example, package name changes).

I also assisted with teaching on multiple occasions, from Bash and Git to a Nvidia deep learning course (see Achievements).
} % Description
%------------------------------------------------
\workposition{Trading Assistant} % Job title
{Sainsbury's} % Employer
{11/2015 -- 09/2017} % Duration
{Goldalming, UK} % Location
{Part Time} % FT/PT (full time or part time)
{Working at Sainsbury's taught me a great deal about working well in a high-pressured,
collaborative environment.
The job required me to be the face of Sainsbury's; interacting with customers to ensure
they have the best shopping experience possible.

From my time at Sainsbury's, I gained skills including social interaction, time management and leadership.

I got on well with my colleagues and often attended the social events organised by Sainsbury's. I believe our productivity was higher because of our ability to work together and support each other.
}  % Description

%------------------------------------------------

\vspace{-\baselineskip}\medskip % Standardise the whitespace after this section and before the next (the custom command adds too much otherwise)

%----------------------------------------------------------------------------------------
%	REFERENCES
%----------------------------------------------------------------------------------------

%\section{References}

%\textit{References available on request}

%------------------------------------------------

% Example \tableentry{} command to add another line:

%\tableentry{Heading}{Content}{spaceafter}

% All 3 parameters must be supplied but any can be empty if you don't need them
% A "spaceafter" value in the third parameter will add some vertical space -- this is to be used between headings

%------------------------------------------------

%\begin{supertabular}{rl} % Start a table with two columns, the table will ensure everything is aligned
	
	%------------------------------------------------
	
%	\tableentry{}{\textbf{Dr. Isaac Kleiner}}{spaceafter}
%	\tableentry{Position}{Professor}{}
%	\tableentry{Employer}{\href{http://web.mit.edu/physics/}{Department of Physics}}{}
%	\tableentry{}{\href{https://web.mit.edu}{\textit{Massachusetts Institute of Technology}}}{spaceafter}
%	\tableentry{Phone}{+1 (617) 253 1000 x5322 (Work)}{}
%	\tableentry{Mobile}{+1 (232) 842-3583}{}
	
	%------------------------------------------------
	
%	\tableentry{}{}{} % Creates some additional whitespace between the references
	
	%------------------------------------------------
	
%	\tableentry{}{\textbf{Dr. Eli Vance}}{spaceafter}
%	\tableentry{Position}{Scientist (HL1)}{}
%	\tableentry{Employer}{\href{http://www.bmrf.us}{Black Mesa Research Facility}}{spaceafter}
%	\tableentry{Email}{\href{mailto:e.vance@bmrf.us}{e.vance@bmrf.us}}{}
%	\tableentry{Phone}{+1 (800) 786-1410 x6235 (Work)}{}
%	\tableentry{Mobile}{+1 (201) 632-3901}{}
	
	%------------------------------------------------
	
%\end{supertabular}

\pagebreak % Bottom of column for page
\switchcolumn % Switch to the next paracol column

%----------------------------------------------------------------------------------------
%	COLOURED CONTACT DETAILS BOX
%----------------------------------------------------------------------------------------

\parbox[top][0.12\textheight][c]{\linewidth}{ % Parbox to hold the colour box; fixed height to match the name/CV text to the left, centred vertically and full line width
	\vspace{-0.04\textheight} % Reduce whitespace above the parbox to separate it from the main content
	\colorbox{shade}{ % Create the coloured box
		\begin{supertabular}{p{0.05\linewidth}|p{0.775\linewidth}} % Start a table with two columns, the table will ensure everything is aligned
			\raisebox{-1pt}{\faHome} & Birmingham, United Kingdom \\ % Address
			\raisebox{-1pt}{\faPhone} & +44 7393 929300 \\ % Phone number
			\raisebox{0pt}{\small\faEnvelope} & \href{mailto:jakefoster954@gmail.com}{jakefoster954@gmail.com} \\ % Email address
			\raisebox{-1pt}{\small\faLinkedin} & \href{https://linkedin.com/in/jake-foster-b433a6173/}{linkedin.com/in/jake-foster-b433a6173/} \\ % Website
			%\raisebox{-1pt}{\small\faGithub} &
			%\href{https://github.com/jakefoster954}{github.com/jakefoster954} \\ % Website
			%\raisebox{-1pt}{\faGithub} & \href{https://github.com/username}{https://github.com/username} \\ % GitHub profile
			%\raisebox{-1pt}{\faLinkedinSquare} & \href{https://www.linkedin.com/in/username}{https://www.linkedin.com/in/username} \\ % LinkedIn profile
			% See fontawesome.pdf in the fonts folder for all icons you can use
		\end{supertabular}
	}
}

%----------------------------------------------------------------------------------------
%	EDUCATION
%----------------------------------------------------------------------------------------

\section{Education} 

% Blank \educationentry{} command to add another degree:

%\educationentry{} % Duration
%{} % Degree
%{} % Honours, achievements or distinctions (e.g. first class honours)
%{} % Department
%{} % Institution

% All 5 parameters must be supplied but any can be empty if you don't need them

%------------------------------------------------

\begin{supertabular}{rl} % Start a table with two columns, the table will ensure everything is aligned

	%------------------------------------------------
	
	\educationentry{2017-Present} % Duration
	{Bachelor of Science (WIP)} % Degree
	{86\% average as of Dec 2020} % Honours, achievements or distinctions (e.g. first class honours)
	{Computer Science} % Department
	{University of Birmingham} % Institution
	
	%------------------------------------------------
	
	\educationentry{2015-2017} % Duration
	{A levels} % Degree
	{A, A, B, C} % Honours, achievements or distinctions (e.g. first class honours)
	{} % Department
	{Godalming College} % Institution
	
	%------------------------------------------------

\end{supertabular}

%----------------------------------------------------------------------------------------
%	TECH SKILLS
%----------------------------------------------------------------------------------------

\section{Tech Skills} 

% Example \tableentry{} command to add another line:

%\tableentry{Heading}{Content}{spaceafter}

% All 3 parameters must be supplied but any can be empty if you don't need them
% A "spaceafter" value in the third parameter will add some vertical space -- this is to be used between headings

%------------------------------------------------

\mybox{Java}
\mybox{Python}
\mybox{C\#}
\mybox{OCaml}
\mybox{Haskell}
\mybox{C}
\mybox{Git} \\
\mybox{LaTeX}
\mybox{Data Science}
\mybox{Machine Learning} \\
\mybox{Reinforcement Learning}
\mybox{Lambda Calculus} \\
\mybox{Networking}
\mybox{Algorithms}
\mybox{SQL}
\mybox{Unity}


%----------------------------------------------------------------------------------------
%	SOFT SKILLS
%----------------------------------------------------------------------------------------

\section{Soft Skills}

% Example \tableentry{} command to add another line:

%\tableentry{Heading}{Content}{spaceafter}

% All 3 parameters must be supplied but any can be empty if you don't need them
% A "spaceafter" value in the third parameter will add some vertical space -- this is to be used between headings

%------------------------------------------------

\mybox{Leadership}
\mybox{Event Management}
\mybox{Technical Writing} \\
\mybox{Time Management}
\mybox{Public Speaking}
\mybox{Mentoring}
\mybox{Problem Solving}
\mybox{Decision Making}

%----------------------------------------------------------------------------------------
%	DISSERTATION
%----------------------------------------------------------------------------------------

\vspace{20pt} % Modify this to make page look nice
\section{Dissertation}
When a person moves to intercept a ball, they first move their head rapidly to focus on the ball.
After this initial movement, they will perform small movements to maintain focus until
the interception point can be predicted with a reasonable degree of accuracy.
Once the intercept has been predicted (Usually around the time the ball reaches the climax of the parabola),
the hands will move to the interception point. All movements have a gradual acceleration and deceleration.

I believe these movement trends occur as a byproduct of noise in the environment. I wish to try to model this process in a reinforcement learning 
environment, providing the agent with minimal, noisy information to see if the agent will obtain motion similar to a human. 

%----------------------------------------------------------------------------------------
%	PUBLICATIONS
%----------------------------------------------------------------------------------------

% As an alternative to a long-form publication list, you can create a shorter summary using only DOI values and years.

% Example \doipublication{} command to add another publication:

%\doipublication{Year}{DOI}{firstauthor}{spaceafter}

% All four parameters are required (can be empty though)
% A value of "firstauthor" in the third parameter will print the DOI in bold
% A "spaceafter" value in the fourth parameter will add some vertical space -- this is to be used between years

%------------------------------------------------

%\begin{supertabular}{rl} % Start a table with two columns, the table will ensure everything is aligned
	
	%------------------------------------------------
	
%	\doipublication{1996}{10.1021/jp951483+}{firstauthor}{spaceafter}
	
	%------------------------------------------------
	
%	\doipublication{1990}{10.1139/p90-097}{firstauthor}{spaceafter}
%	\doipublication{1986}{10.1139/v86-297}{}{}
	
	%------------------------------------------------
	
%	\doipublication{1986}{10.1103/PhysRevA.34.2329}{}{spaceafter}
	
	%------------------------------------------------
	
%	& \textit{First author publications in} \textbf{bold}\\
	
%	%------------------------------------------------
	
%\end{supertabular}

% \medskip % Extra whitespace before the next section

%----------------------------------------------------------------------------------------
\pagebreak
\switchcolumn % Switch to the next paracol column
%----------------------------------------------------------------------------------------


%----------------------------------------------------------------------------------------
%	Achievements
%----------------------------------------------------------------------------------------

\section{Achievements}
\workposition{CSS UoB Treasurer} % Job title
{} % Employer
{03/2020-03/2021} % Duration
{} % Location
{} % FT/PT (full time or part time)
{I am the treasurer of the award winning departmental society for computer science at the university.
We have hosted a number of events, including but not limited to online game nights and pub quizzes.

My role as treasurer requires me to manage the finances, book venues, authorise bank transfers and assist with
the acquisition of sponsors.
}  % Description

\workposition{BCS Committee Member} % Job title
{} % Employer
{11/2018-10/2022} % Duration
{} % Location
{} % FT/PT (full time or part time)
{I was co-opted onto the BCS Brimingham branch committee November 2018, then was officially
elected October 2019 to serve for 3 years.
In October 2020 I became the Student Outreach Officer, tasked with bringing awareness of the BCS to students at UoB, Aston and BCU.

As a committee member, I assist with the organisation of events. I have led a debate for the annual AGM
and have promoted BCS in front of an audience consisting of hundreds of people.

I have represented the organisation as a sponsor at Hack the Midlands, a 24hr creative marathon on multiple occasions.
}  % Description

\workposition{Google Hash code 2019} % Job title
{} % Employer
{02/2019} % Duration
{} % Location
{} % FT/PT (full time or part time)
{I collaborated with 3 friends in Google Hash code. We were called Team8472 and we achieved a score
of 853560 putting us 444th in the world and top 20 in the UK. More than 70,000 students and professionals
from around the world registered for Hash Code 2019.
}  % Description

\workposition{Teaching Assistant} % Job title
{} % Employer
{07/2019-08/2020} % Duration
{} % Location
{} % FT/PT (full time or part time)
{As part of my Industrial Placement, I was required to assist with teaching seminars on
Bash, Git, Python, Slurm and Nvidia deep learning tools using the HPC Cluster.

The sessions were covered over 1-2 days and involved me answering questions and assisting with problems.
}  % Description

\workposition{Capture the Flags} % Job title
{} % Employer
{03/2019} % Duration
{} % Location
{} % FT/PT (full time or part time)
{I take part in the annual BAE Systems Capture the Flag that is held at the University of Birmingham.

A "CTF" is a hacking competition where you obtain flags from vulnerable applications.

I placed 2nd in the 2019 CTF, missing out on first place by a couple hundred points. CTF's are a great way to practice useful
security skills like Buffer Overflow attacks, Reverse Engineering attacks, Stenography, SQL Injection etc.
}  % Description

%----------------------------------------------------------------------------------------

\pagebreak % Bottom of column for page
\switchcolumn % Switch to the next paracol column

%----------------------------------------------------------------------------------------
%	Projects
%----------------------------------------------------------------------------------------

\section{Projects}
\workposition{Game Jams} % Job title
{Game Development Society} % Employer
{03/2018, 10/2018, 03/2019} % Duration
{} % Location
{} % FT/PT (full time or part time)
{I have participated in multiple game development competitions where I am
required to create a game within a 48hr period. I created my own game engine in Java which I
originally used to develop these games. The first game I made was a side-scrolling platformer. The full game
was never completed however a demo level was created as a prototype and all the key mechanics were functional.

More recently, I have created a couple 2D games in Unity (Code written in C\#).}  % Description

%----------------------------------------------------------------------------------------

\workposition{Hackathons} % Job title
{Hack the Midlands} % Employer
{10/2018} % Duration
{} % Location
{} % FT/PT (full time or part time)
{Hack the Midlands is a 24hr creative marathon. I was a participant in 2018,
then attended as a representative of the BCS Birmingham branch in 2019 and 2020.
I experimented with Amazon Alexa and a range of different API’s to create a
program that could play naughts and crosses (theoretically online) through Alexa.
Getting to see what the other participants had created was fascinating and gave me a lot of ideas about future projects I would like to work on.}  % Description

%----------------------------------------------------------------------------------------

\workposition{} % Job title
{IBM QuisKit} % Employer
{11/2019} % Duration
{} % Location
{} % FT/PT (full time or part time)
{I have also attended a quantum computing Hackathon designed around using Quizkit - an open source
framework for quantum computing.}  % Description

%----------------------------------------------------------------------------------------

%----------------------------------------------------------------------------------------
%	Personal Interests
%----------------------------------------------------------------------------------------

\section{Personal Interests}
\longformdescription{Board Games}
{I often play board games with my housemates. I am a competitive player and enjoy theory crafting 
strategies to create an optimal set of moves.}

\longformdescription{Snooker}
{I grew up playing pool and snooker down my local pub. I gave it up when I came to university but
would love the opportunity to return to the table.}

\longformdescription{Dungeons and Dragons}
{DnD is a great opportunity to get to act in a way you would not normally act. It is a chance to
avoid the stresses of everyday life for a short time while you are your character.}
%----------------------------------------------------------------------------------------
\end{paracol}

%----------------------------------------------------------------------------------------
\end{document}
